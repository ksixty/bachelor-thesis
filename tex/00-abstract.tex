% Также можно использовать \Referat, как в оригинале

\pagestyle{empty}

\chapter*{Аннотация}

Наименование работы: <<Программная среда для проведения дистанционных испытаний в области защиты информации>>.

Цель работы: спроектировать и разработать среду для проведения дистанционных испытаний в области защиты информации.

Объект исследования работы --- программно-аппаратная среда для проведения всероссийской школьной олимпиады по защите информации Ugra CTF School.

Работа состоит из введения, трёх глав, заключения и приложений. Общий объём работы --- \pageref{LastPage}, из них 2 приложения на 63 листах. В работу включены 14 рисунков, 6 таблиц. Библиографический список состоит из 75 источников: 37 электронных ресурсов и 38 печатных текстов.

Введение содержит обоснование актуальности выбранной темы, цель и задачи аттестационной работы.

Первая глава содержит историческую справку о соревнованиях в области защиты информации, правила и принципы их проведения, сведения об олимпиаде Ugra CTF и её организационных особенностях. В этой главе вводится понятие CTF-соревнований и распределённо-очного формата проведения школьных олимпиад.

Вторая глава посвящена изучению организационных, технических и правовых требований, предъявляемых к проведению олимпиады Ugra CTF и теоретическим изысканиям, которые связаны с проектированием среды для проведения этой олимпиады.

В третьей главе изложены сведения об итоговой архитектуре автоматизированной системы, обеспечивающей создание и управление средой для проведения испытаний, а также технические детали её реализации.

В заключении формулируются выводы о проделанной работе и описывается ход апробации разработанной системы на практике, в реальных условиях олимпиады Ugra CTF School 2022, которая прошла 2 апреля 2022 года одновременно в 10 городах России.

\pagestyle{empty}
\chapter*{Abstract}

Title: ``Software environment for conducting remote information security challenges''.

The object of the research is a computer environment for all-Russian school Olympiad on information security, Ugra CTF School.

The work consists of an introduction, three chapters, a conclusion and appendices. This work spans \pageref{LastPage} pages, 63 of which are the 2 appendices. The paper comprises 14 figures, 6 tables. Bibliography list comprises 75 sources: 37 electronic resources and 38 printed texts.

In the introduction we express the rationale for picking this topic, and state the purpose and objectives of this work.

In the first chapter we review the historical background of the information security competitions, their rules, and organizational nuances of Ugra CTF Olympiad. This chapter introduces the notion of a CTF competition and the distributed format of the school Olympiads.

In the second chapter we study organizational, technical and legal requirements for conducting the Ugra CTF --- and explore theoretical questions involved with the design of an automated system for providing a computer environment for participants to solve Olympiad's challenges.

In the third chapter we review the final architecture of the system, as well as the details of its implementation.

In the conclusion we analyze the results and describe the progress of testing the developed system at a real event, during the Ugra CTF School 2022 Olympiad, which was held on April 2, 2022 simultaneously in 10 cities of Russia.

%%% Local Variables:
%%% mode: latex
%%% TeX-master: "rpz"
%%% End:
