%\NirEkz{Экз. 3}                                  % Раскоментировать если не требуется
%\NirGrif{Секретно}                % Наименование грифа

\gosttitle{Gost7-32}       % Шаблон титульной страницы, по умолчанию будет ГОСТ 7.32-2001,
% Варианты GostRV15-110 или Gost7-32

\NirOrgLongName{Министерство науки и высшего образования Российской Федерации \\
РОССИЙСКОЙ ФЕДЕРАЦИИ\\[1em]
Федеральное государственное автономное учреждение высшего образования\\
<<МОСКОВСКИЙ ПОЛИТЕХНИЧЕСКИЙ УНИВЕРСИТЕТ>>
}                                           %% Полное название организации

\NirUdk{ }
\NirGosNo{ }
\NirInventarNo{ }

\NirConfirm{Допущен к защите}                  % Смена УТВЕРЖДАЮ
\NirBoss[.49]{И.о. заведующего кафедрой,\\руководитель образовательной программы}{/ А. Ю. Гневшев /}            %% Заказчик, утверждающий НИР

\NirReportName{Выпускная квалификационная работа}   % Можно поменять тип отчета
\NirAbout{} %Можно изменить о чем отчет

%\NirPartNum{Часть}{1}                      % Часть номер

\NirBareSubject{}                  % Убирает по теме если раскоментить

\Nir{по направлению \par
        10.03.01 Информационная безопасность \par
        (бакалавр) \par
        Образовательная программа (профиль) \par
        <<Безопасность компьютерных систем>> \par
        на тему:}

\NirSubject{<<Проектирование и разработка среды для проведения дистанционных испытаний в области защиты информации>>}                                   % Наименование темы
\NirFinal{}                        % Заключительный, если закоментировать то промежуточный
\finalname{черновая}               %п Название финального отчета (Заключительный)
%\NirCode{Шифр\,---\,САПР-РЛС-ФИЗТЕХ-1} % Можно задать шифр как в ГОСТ 15.110
\NirCode{}

\NirIsp{Студент}{Иван Сергеевич Клименко, 181-351} %% Название руководителя
\NirManager{Руководитель ВКР}{Александр Сергеевич Красников, к.ф-м.н.} %% Название руководителя

\NirYear{2022}%% если нужно поменять год отчёта; если закомментировано, ставится текущий год
\NirTown{Москва ---}                           %% город, в котором написан отчёт
