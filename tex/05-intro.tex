\Introduction

%% Всюду компьютеры. Уже выученная наизусть реклама на экранах вагонов метро. Автомобили, которыми больше не нужно управлять человеку. Светофоры, через которые они проносятся. Стремительность распространения технологий в мире вокруг нас прямо пропорциональна неожиданности этого явления\cite{DigitalizatonTempo}. Так же пропорционален рост желания большинства\cite{DigitalizationTempo2} оцифровать свою жизнь --- и темпы, с которыми появляются новые риски и угрозы для этогой новой оцифрованной жизни: кибератаки, монополии и неполное понимание устройства технологий, которые окружают человека сегодня. Не сходится с этой динамикой экспоненциального роста лишь уровень осведомлённости о кибербезопасности.

Обучение кибербезопасности --- актуальный вопрос, важный для автора данной работы. С 2017 года он так или иначе связан с этой деятельностью, помогая разрабатывать и проводить соревнования по защите информации в формате CTF \textit{(англ. capture the flag --- «захват флага»)}, а также разрабатывая методики преподавания сопутствующего материала\cite{NTISchool} и способствуя популяризации этой темы среди школьников России. При текущих трендах цифровизации и развития техники будущим поколениям придётся столкнуться с куда более серьёзными вызовами и сложностями, о которых сегодня можно лишь догадываться: например, в последнее время более активно, чем прежде, проводятся исследования в области квантовых вычислений\cite{Quantum1}\cite{Quantum2}, успех в которых гарантированно приведёт к утрате стойкости всех распространённых криптографических методов\cite{Quantum3}.

Один из способов повышения осведомлённости о кибербезопасности --- внедрение её азов в формате школьных олимпиад. Проведение олимпиад по защите информации по правилам Российского совета олимпиад школьников --- непростая задача, требующая принципиально нового подхода как к организационной, так и к технической подготовке.

Формат CTF, в котором, как правило, проводятся подобные мероприятия, накладывает большое количество требований к рабочим станциям, на которых участники должны выполнять задания (возможность устанавливать дополнительное ПО, изменять конфигурацию системы и использовать произвольную ОС). Кроме того, для решения CTF-задач участникам необходим доступ в интернет --- это не противоречит правилам РСОШ, если участники не используют сеть для общения, но зафиксировать нарушешие правил олимпиады становится на порядок сложнее.

Организация, практику в которой проходит автор настоящей работы, проводит соревнования на независимых площадках с наблюдателями-волонтёрами, квалификация которых варьируется и не всегда достаточна для того, чтобы гарантировать соблюдение всех требований по проведению подобных мероприятий. Если задачу возможно автоматизировать или успростить, то лучше так и поступить.

Цель данной выпускной квалификационной работы --- спроектировать и разработать среды для проведения дистанционных испытаний в области защиты информации, для чего необходимо решить следующие задачи:
\begin{itemize}
\item определить требования к такой системе, основываясь на специфике CTF-соревнований, требованиях РСОШ и технических возможностях оргкомитета олимпиады;
\item выбрать лучшее из доступных готовых решений, если оно удовлетворяет всем требованиям, обосновать выбор;
\item изучить технологии, позволяющие достичь требований без готовых решений либо с их доработкой в противном случае;
\item разработать проект архитектуры среды для проведения испытаний;
\item разработать саму систему;
\item опробовать её в ходе очного этапа соревнований по защите информации Ugra CTF School, который состоится 2 апреля 2022 года;
\end{itemize}
