\Introduction
Безопасность информации — актуальный вопрос, поскольку по мере роста значимости киберпространства и развития технологий, растёт и серьёзность угроз, которые в нём возникают. При текущих треднах будущим поколениям придётся столкнуться с куда более серьёзными вызовами и сложностями в этой сфере, о которых сегодня можно лишь догадываться: например, в последнее время более активно, чем прежде, проводятся исследования в области квантовых вычислений\cite{Quantum1}\cite{Quantum2}, успех в которых гарантированно приведёт к утрате стойкости всех распространённых криптографических методов\cite{Quantum3}.

Один из способов повышения осведомлённости о проблемах кибербезопасности — внедрение её азов в формате школьных олимпиад.

С 2016 года в России проводится школьная олимпиада по защите информации Ugra CTF\cite{UgraHistory}. За её организацию отвечает Югорский научно-исследовательский институт информационных технологий, практику в котором проходит автор настоящей работы.

Проведение мероприятий подобного уровня, с одной стороны, требует полного соблюдения требований Российского совета олимпиад школьников\cite{Rosolymp}, которые в строгом порядке регламентируют действия как участников, так и организаторов. С другой стороны, для этого необходимо создать участникам максимально приближенные к настоящим условия для демонстрации своих навыков в сфере практической кибербезопасности — по этой причине они проводятся в формате CTF \textit{(англ. capture the flag, «захват флага»)}. Формат игровой: участники исследуют модели компьютерных систем на предмет уязвимостей, эксплуатация либо устранение которых приносит очки.

Основная сложность заключается в том, что заключительный этап олимпиады --- очный и проводится более чем в десяти городах по всей стране\cite{Statforma}. Организационно-технический процесс, включающий в себя подготовку рабочих мест участников, настройку инфраструктуры и дистанционный контроль за ходом мероприятия на разных площадках, можно сделать эффективнее, разработав и внедрив в него элементы автоматизации.

Цель данной выпускной квалификационной работы --- спроектировать и разработать среду для проведения дистанционных испытаний в области защиты информации.

Проектирование и разработка такой среды достигается решением следующих задач:
\begin{itemize}
\item изучить особенности проведения испытаний в формате: требования РСОШ, принципы и правила CTF-соревнований;
\item разработать перечень технических требований к среде для проведения дистанционных испытаний;
\item спроектировать архитектуру среды и описать принципы её работы;
\item создать автоматизированную систему (АС) для управления такой средой.
\Abbrev{АС}{автоматизированная система}
\end{itemize}
