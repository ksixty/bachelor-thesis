\Introduction

Обучение кибербезопасности --- актуальный в обществе вопрос\cite{DigitalisationTempo}. По мере роста значимости киберпространства растёт и серьёзность угроз, которые в нём возникают. Обеспечение защиты информации — сложная задача уже сегодня, и над её решением трудится большое количество людей из разных сфер жизни: учёные и инженеры, юристы и законодатели.

При текущих трендах цифровизации и развития техники будущим поколениям придётся столкнуться с куда более серьёзными вызовами и сложностями, о которых сегодня можно лишь догадываться: например, в последнее время более активно, чем прежде, проводятся исследования в области квантовых вычислений\cite{Quantum1}\cite{Quantum2}, успех в которых гарантированно приведёт к утрате стойкости всех распространённых криптографических методов\cite{Quantum3}.

В России с 2016 года проводится школьная олимпиада по защите информации Ugra CTF\cite{UgraHistory}. Она организована при Югорском научно-исследовательском институте информационных технологий, практику в котором проходит автор настоящей работы.

Формат CTF \textit{(англ. capture the flag, «захват флага»)}, в котором, как правила, проодятся соревнования в области кибербезопасности, подразумевает исследование участниками компьютерных систем на предмет уязвимостей, корректная эксплуатация которых приносит игровые очки. Проведение олимпиады в формате CTF — сложная задача. Её решение, с одной стороны, требует полного соблюдения требований Российского совета олимпиад школьников\cite{Rosolymp}, а с другой — наличия достаточной компьютерной инфраструктуры, позволяющей проводить подобные соревнования.

Олимпиада Ugra CTF проводится на независимых площадках в разных городах с наблюдателями-волонтёрами, квалификация которых варьируется и не всегда достаточна для того, чтобы гарантировать соблюдение всех требований по проведению подобных мероприятий. Если задачу возможно автоматизировать или успростить, то лучше так и поступить.

Цель данной выпускной квалификационной работы --- спроектировать и разработать среды для проведения дистанционных испытаний в области защиты информации, для чего необходимо решить следующие задачи:
\begin{itemize}
\item определить требования к такой системе, основываясь на специфике CTF-соревнований, требованиях РСОШ и технических возможностях оргкомитета олимпиады;
\item разработать проект архитектуры среды для проведения испытаний;
\item разработать саму систему.
% \item опробовать её в ходе очного этапа соревнований по защите информации Ugra CTF School, который состоялся 2 апреля 2022 года;
\end{itemize}
